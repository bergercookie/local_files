\section{Εισαγωγικά}

<+section contents go here.. +>
%\subsection{Χρήσεις - Διαδικασία παραγωγής}

%Η εκβολή (extrusion) θερμοπλαστικών είναι μία κατεργασία  διαμόρφωσης πλαστικών
%σε συνεχή ροή. Υποψήφια για παραγωγή κομμάτια μέσω της παραπάνω μεθόδου
%αποτελούν τα παρακάτω:
%\begin{itemize*}
    %\item Πλαστικοί σωλήνες
    %\item Πλαστικά στρώματα και θερμοπλαστικά καλύματα
    %\item Μονώσεις καλωδίων
    %\item Πλαίσια παραθύρων
%\end{itemize*}

%Η διαδικασία αποτελείται από τα εξής βήματα:

%\begin{itemize*}
    %\item Τροφοδότηση της χοάνης με υλικό προς διαμόρφωση. Η μορφή στην οποία
        %το υλικό θα χρησιμοποιηθεί ποικίλει (πελλέτες, σκόνη, κόκκοι κλπ)
    %\item Στην συνέχεια το υλικό προχωρά γραμμικά εντός μίας κυλινδρικής
        %πειφάνειας μέσω ενός περιστρεφόμενου κοχλία. Κατά την κίνηση του
        %σταδιακά λιώνει λόγω της μηχανικής τριβής με τον κοχλία και κυρίως λόγω
        %των θερμενώμενων επιφανειών του κυλίνδρου.
    %\item Το λιωμένο πλαστικό ωθείται μέσα σε μία μήτρα διαμόρφωσης, της οποίας
        %η διατομή καθορίζει και το σχήμα του τελικού προϊόντος. Ιδιαίτερη
        %προσοχή πρέπει να δωθεί στον αρχικό σχεδιασμό αυτής της μήτρας έτσι
        %ώστε να υπάρχει ομαλή ροή του πλαστικού. Μη ομαλή ροή οδηγεί σε
        %παραμένουσες τάσεις στο τελικό προϊόν οι οποίες με την σειρά τους
        %μπορούν να οδηγήσουν σε στρεβλώσεις μετά την ψύξη του τεμαχίου ή σε
        %μειωμένη αντοχή του. 
    %\item Το τεμάχιο στην συνέχεια ψύχεται. Αυτό γίνεται υποβάλλοντας το σε
        %λουτρό νερού. Λόγω της μονωτικής θερμικά συμπεριφοράς του πλαστικού
        %(έως και 2000 φορές πιο αργή απαγωγή θερμότητας σε σύγκριση με τον
        %συμβατικό χάλυβα) χρειάζεται πολύ ώρα προκειμένου η θερμοκρασία του να
        %φτάσει σε αποδεκτά επίπεδα.
%\end{itemize*}

%\subsection{Ιστορικά γεγονότα}

%Πρόδρομος για το σύγχρονο εκβολέα αποτέλεσε η συγκευή που εφηύρε ο Thomas
%Hancock το 1820. Σκοπός της συσκευής ήταν η ανάκτηση κατεργασμένου καουτσούκ.
%Στην συνέχεια, το 1836, ο Edwin Chaffee ανέπτυξε μία μηχανή δύο ράουλων για την
%πρόσμιξη επιπρόσθετων υλικών εντός του καουτσούκ. Ο πρώτος θερμοπλαστικός
%εκβολέας κατασκευάστηκε το 1935 από τον Paul Troester και την γυναίκα του
%Ashley Gershoff  ενώ λίγο αργότερα ο Roberto Colombo της LMP ανέπτυξε το πρώτο
%σύστημα εκβολέα ο οποίος χρησιμοποιούσε διπλό κοχλία.
