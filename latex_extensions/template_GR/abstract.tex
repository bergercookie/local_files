\begin{abstract}
Σκοπος της συγκεκριμένης αναφοράς είναι η επεξεργασία δωσμένων μετρήσεων μηχανών
και η διάγνωση βλαβών που πιθανώς υπάρχουν σε αυτές. Πιο συγκεκριμένα δίνεται
βιομηχανική εγκατάσταση στην οποία μέσω επιταχυνσιόμετρου μετράμε τους
κραδασμούς σε συγκεκριμένες χρονικές στιγμές. Με βάση αυτά τα δεδομένα και για
δύο διαφορετικές περιπτώσεις λειτουργίας της μηχανής εκτελούμε ανάλυση φάσματος
και χρονικής κυματομορφής και διαπιστώνουμε κατα πόσο η μηχανή δουλεύει
ικανοποιητικά ή παρουσιάζει κάποια συγκεκριμένη βλάβη. Τέλος όταν
κρίνεται απαραίτητο εκτελούμε αποδιαμόρφωση του σήματος και προχωρούμε σε
περεταίρω ανάλυση του. 

Μετά από ανάλυση φάσματος και κυματομορφής των εκάστοτε μετρήσεων καταλήγουμε στο
συμπέρασμα πως στην πρώτη περίπτωση επικρατεί η φθορά στον εσωτερικό δακτύλιο
ενός εκ των ρουλεμάν στις θέσεις Α,Β καθώς και ένα (μικρότερης έκτασης) πρόβλημα
κακής ευθγράμμισης του άξονα. Στη δεύτερη περίπτωση επικρατεί το πρόβλημα της
φθοράς εξωτερικού δακτυλίου στα ρουλεμάν των θέσεων Α,Β. 

Η αναφορά προορίζεται για την εξαμηνιαία εργασία στο μάθημα «Δυναμική Μηχανών
ΙΙ» του 7ου εξαμήνου της σχολής Μηχανολόγων Μηχανικών του Ε.Μ.Π.
Η ανάλυση των σημάτων καθώς και η παραγωγή των απαραίτητων διαγραμμάτων γίνεται
μέσω της τεχνικής γλώσσας προγραμματισμού MATLAΒ.
\end{abstract}

% Add nomenclature here.. 
% run supplementary2latex afterwards

%\nomenclature{$<+BPFO+>$}{<+Χαρακτηριστική συχνότητα φθοράς εξωτερικού δακτυλίου+>}%

